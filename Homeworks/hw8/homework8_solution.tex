\documentclass[12pt]{article}
\usepackage[english]{babel}
\usepackage{amsmath}
\usepackage{amsthm}
\usepackage{amssymb}
\usepackage{amsfonts}
\title{HDP Notes and Exercise Soultions}
\author{Holden Caulfield}
\newtheorem{theorem}{Theorem}
\newtheorem{remark}{Remark}[theorem]
\newtheorem{ex}{Problem}
\usepackage{enumitem}
\usepackage{graphicx}
\usepackage{tcolorbox}
\usepackage{caption}
\usepackage{xcolor}
\usepackage{tikz}
\tcbuselibrary{skins}
\tcbuselibrary{breakable}

\newtcolorbox{bx}[1]{skin=bicolor,breakable,leftrule=1mm,toprule=0mm,bottomrule=0mm,rightrule=0mm,colbacklower=red!15,colback=gray!15,sharp corners,colframe=red}

\begin{document}
	\centering	\section*{Homework 8 Solution}
	
	\begin{bx}
		
		\begin{ex}
		\;
			\begin{enumerate}[label=(\alph*)] 
				
			\item Compute the operator and Frobenius norms of the identity matrix.
			\item Compute the operator and Frobenius norms of the matrix whose all entries = 1.
				
			\end{enumerate}
		\end{ex}
		\tcblower
		\begin{enumerate}[label=(\alph*)]
		\item $\|I\|_F=n$,$\|I\|=1$
		\item $\|A\|_F=n^2$,$\|A\|=n$
		 \end{enumerate}
		\end{bx}
	
	
		\begin{bx}
		
		\begin{ex}
				Let A be an $n\times n$ symmetric matrix.
			\begin{enumerate}[label=(\alph*)]
				\item Show that
				\[
				\|A\|=\max\limits_{i=1,\dots,n}|\lambda_i(A)|
				\]
				where $\lambda_i(A)$ denote the eigenvalues of A.
				
				\item Show that
				$\|A\|=\max\limits_{x\in S^{n-1}}|x^\top Ax|$
				
				\item Show by example that the formula in (b) may fail for non-symmetric matrices.
			\end{enumerate}
		\end{ex}
		\tcblower
	\begin{enumerate}[label=(\alph*)]
		\item We can write the spectral decomposition of $A$ since $A$ is self-adjoint:
		\[
		A=\sum_{i=1}^{n}\lambda_iu_iu_i^\top
		\]
		\[
		AA^\top=\sum_{i=1}^{n}\lambda_i^2u_iu_i^\top
		\]
		which implies that
		$\sigma_i=|\lambda_i|$, so:
		\[
		\| A \| = \max\limits_{i=1,\dots,n}|\lambda_i(A)|
		\]
		
		\item We can rewrite the objective as $\langle x,Ax\rangle$. Knowing that $A$ is symmetric, we have a orthonormal basis of eigenvectors of $A$ due to the real spectral theorem. We can then write $x=\sum_{i=1}^{n}\langle x,u_i\rangle u_i$
		where $u_i$'s form the aforementioned basis.
		Thus, we can write the objective as:
		\begin{align*}
	|	\langle A\sum_{i=1}^{n}\langle x,u_i\rangle u_i,\sum_{i=1}^{n}\langle x,u_i\rangle u_i\rangle| &= 
	|	\langle \sum_{i=1}^{n}\lambda_i\langle x,u_i\rangle u_i,\sum_{i=1}^{n}\langle x,u_i\rangle u_i\rangle |\\&=|\sum_{i=1}^{n}\langle x,u_i \rangle^2\lambda_i| \\
	&\le |\lambda_{max}||\sum_{i=1}^{n}\langle x,u_i\rangle^2| \\
	&= |\lambda_{max}|
		\end{align*}
So we have:
\[
\max\limits_{x\in S^{n-1}}x^\top Ax=\max\limits_{i=1,\dots,n}|\lambda_i(A)|=\|A\|
\]
where the second equality is the result of part (a).

\item Consider the matrix:
$$
A =
\begin{bmatrix}
	0 & 8 \\
	2 & 0
\end{bmatrix}
$$
we have:
\[
AA^\top = \begin{bmatrix}
	64 & 0 \\
	0 & 4
\end{bmatrix}
\]
so $\|A\|=8$.

Now consider a unit vector $u=[u_1,u_2]^\top$, then $u^\top A u=10u_1u_2$, we know that $u_1^2+u_2^2=1$, so $u_1u_2\le\frac{1}{2}$ (this is easy to check using  Lagrangian dual), which implies that the relation given in part (b) would give us $5$ as the answer, which is not equal to $\|A\|=8$.
	\end{enumerate}
\qed
	\end{bx}
	
		\begin{bx}
		
		\begin{ex}
			Let $X_1,X_2$ be independent $N(0,1)$ random
			variables. Show that $Y_1=(X_1+X_2)/\sqrt{2}$ and $Y_2=(X_1-X_2)/\sqrt{2}$ are independent $N(0,1)$ random variables.
		\end{ex}

		\tcblower
	Let $X=(X_1,X_2)\sim \mathcal{N}(0,I)$ and $Y=(Y_1,Y_2)$, then $Y=AX$, where:
	\[
	A =
	\begin{bmatrix}
		\frac{1}{\sqrt{2}} & 	\frac{1}{\sqrt{2}} \\
			\frac{1}{\sqrt{2}} & 	-\frac{1}{\sqrt{2}}
	\end{bmatrix}
	\]	
	we know that $Cov(Y)=AA^\top=I$, which implies that $Y_1$ and $Y_2$ are jointly normal independent random variables with variance 1.
	
	It is also easy to check that $Y_1$ and $Y_2$ have mean 0 using the linearity of expected value. 
	
	Note that 0 covariance does not imply independence in general, but it does in the case where the variables are jointly normal, which holds in our case.
	\end{bx}

	
\begin{bx}
	
	\begin{ex}
	Let $A$ be an $n \times n$ symmetric matrix, $\varepsilon \in (0,1/2)$, and N be an $\varepsilon$-net of the unit sphere	$S^{n-1}$. Show that
	\[
	\|A\|\le\frac{1}{1-2\varepsilon}\cdot\max\limits_{x\in\mathcal{N}}|x^\top Ax|
	\]
	\end{ex}
	\tcblower
	Assume that $\|A\|=x^\top A x$ for some $x \in S^{n-1}$. There exists some $y\in\mathcal{N}$ s.t. $\|x-y\|_2\le \varepsilon$. We have:
	\begin{align*}
		|x^\top Ax|-|y^\top Ay| &\le |x^\top Ax-y^\top Ay|\\
		&= |x^\top Ax - x^\top Ay + x^\top Ay-y^\top Ay| \\
		&\le  |x^\top Ax - x^\top Ay| + |x^\top Ay-y^\top Ay| \\
		&= |\langle x, A(x-y)\rangle| +  |\langle x-y,Ay\rangle| \\
		&\le 2\|A\|\varepsilon \\
		&\Rightarrow \|A\|-2\varepsilon\|A\| \le |y^\top A y| \\
		&\Rightarrow \|A\| \le \frac{1}{1-2\varepsilon}\cdot\max\limits_{x\in \mathcal{N}}|x^\top A x|
	\end{align*}

\end{bx}


\end{document}