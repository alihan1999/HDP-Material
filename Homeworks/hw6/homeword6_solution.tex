\documentclass[12pt]{article}
\usepackage[english]{babel}
\usepackage{amsmath}
\usepackage{amsthm}
\usepackage{amssymb}
\usepackage{amsfonts}
\title{HDP Notes and Exercise Soultions}
\author{Holden Caulfield}
\newtheorem{theorem}{Theorem}
\newtheorem{remark}{Remark}[theorem]
\newtheorem{ex}{Problem}
\usepackage{enumitem}
\usepackage{graphicx}
\usepackage{tcolorbox}
\usepackage{caption}
\usepackage{xcolor}
\usepackage{tikz}
\tcbuselibrary{skins}
\tcbuselibrary{breakable}

\newtcolorbox{bx}[1]{skin=bicolor,breakable,leftrule=1mm,toprule=0mm,bottomrule=0mm,rightrule=0mm,colbacklower=red!15,colback=gray!15,sharp corners,colframe=red}

\begin{document}
\centering	\section*{Homework 6 Solution}

\begin{bx}
	
	\begin{ex}
	Prove the following:
	\begin{enumerate}[label=(\alph*)]
		\item Let $X=(X_1,\dots,X_n)$ and $Y=(Y_1,\dots,Y_n)$ be independent symmetric Bernoulli random vectors in $\mathbb{R}^n$, i.e. all coordinates $X_i$ and $Y_i$ areindependent random
		variables that take values ±1 with probability 1/2. Show that:
		\[
		\mathbb{P}\{\left | \langle X,Y\rangle \right | \ge 0.001n\} \le 2exp(-cn)
		\]
	\item Deduce that the angle $\angle(X,Y)$ between the vectors X and Y satisfies
		\[
	\mathbb{P}\{\left | \angle (X,Y) - \frac{\pi}{2} \right | \ge 0.01\pi\} \le 2exp(-c_1n)
	\]
	\item Prove that for every dimension $n > C$, there exist $N\ge \frac{1}{2} exp(cn)$ vectors $v_1,\dots,v_N$
	in $\mathbb{R}^n$ so that all pairwise angles between these vectors are between $89^\circ$ and $91^\circ$.
		\end{enumerate}
	\end{ex}
	\tcblower
\begin{enumerate}[label=(\alph*)]
	\item for every $i$, $X_iY_i$ is a symmetric Bernoulli random variable, and the inner product can be written as a sum of independent symmetric Bernoulli random variables.Applying Hoeffding's inequality:
	$$	\mathbb{P}\{\left |\sum_{i=1}^{n}X_iY_i \right | \ge 0.001n\} \le 2exp(-5.10^{-7}n)$$.
	\item Denoting the angle with $\theta$, $\cos\theta
	 =\frac{\langle X,Y \rangle}{n}$.
	 Further
	 \begin{align*}
	 	\mathbb{P}\{\left | \theta - \frac{\pi}{2} \right | \ge 0.01\pi\}&=\mathbb{P}\{\left |\cos\theta\right | \ge \sin(0.01\pi)\} \\
	 	&= \mathbb{P}\{\left |\langle X,Y\rangle \right | \ge n\sin(0.01\pi)\} \\
	 	&\le 2exp(-c_1n) \; \; (Hoeffding's)
	 \end{align*}
 \item Denoting the angle between every pair of vectors as $\theta_{ij}$.
 \[
 \mathbb{P}\{\left | \theta_{ij} - \frac{\pi}{2} \right |\ge \frac{\pi}{180}\} \le 2exp(c^{\prime}n)
 \]
 Thus the probability of existing a pair with an angle not lying in the given interval is bounded above by $2exp(-c^{\prime}n)N^2$ using union bound.
 We want at least $\frac{exp(cn)}{2}$ such vectors.
 Setting an upper bound of $0.01$, we want
 \[
 exp(n(2c-c^{\prime})) \le 2.10^{-2}
 \]
 picking $c < \frac{c^\prime}{2}$, the inequality holds for $n>C$ for some $C$ since the exponential function is not bounded.
\end{enumerate}
\qed
\end{bx}

\begin{bx}
	
	\begin{ex}
		Prove the following:
		\begin{enumerate}[label=(\alph*)]
			\item Let $z$ be a fixed unit vector in $\mathbb{R}^d$. Show that the $ X = Gz$ is a random vector in $\mathbb{R}^n$ whose all coordinates $X_j$ are independent random variables, which satisfy
			\[
			\mathbb{E}X_j=0 ,\;\;\; Var(X_j)=1 , \;\;\; \|X_j\|_{\psi_2} \le C_1
			\]
			\item  Prove a thin-shell inequality for $X = Gz$:
			\[
			\mathbb{P}\{0.99\sqrt{n} \le \|X\|_2 \le 1.01 \sqrt{n}\} \ge 1-2exp(-cn)
			\]
			\item Let $x_1,\dots,x_N$ be any set of fixed vectors in $\mathbb{R}^d$.
			Let $G$ be an $n \times d$ Bernoulli random matrix, and set $T = \frac{G}{\sqrt{n}}$. Prove that if $n=ClogN$, then the map $T$ approximately preserves the pairwise geometry of the data, namely that following event holds with positive probability:
			\[
			0.99\|x_i-x_j\|_2 \le \|Tx_i-Tx_j\|_2 \le 1.01\|x_i-x_j\|_2 \;\; \; \text{for all } i,j \in [N]
			\]
		\end{enumerate}
	\end{ex}
	\tcblower
	\begin{enumerate}[label=(\alph*)]
	\item Denoting the ith row of $G$ with $G_i$
	\[
	\mathbb{E}X_j = \mathbb{E}G_jz = \sum_{i=1}^{n}\mathbb{E}G_{ji}z_i = 0
	\]
	\[
	Var(X_j) = \mathbb{E}X_j^2 = \sum_{k,i=1}^{n}\mathbb{E}[G_{ji}G_{jk}]z_iz_k = \sum_{i=1}^{n}z_i^2=1 \; \; (unit)
	\]
		For the third one, we know that $X_j$ is a sum of independent symmetric Bernoulli random variables with coefficients $z_i$, thus we can apply Hoeffding's inequality, which yields
	\[
	\mathbb{P}\{\left | G_jz\right | \ge t\} \le 2exp(\frac{-t^2}{2\|z\|_2^2}) = 2exp(\frac{-t^2}{2})
	\]
	which implies that $\|X_j\|_{\psi_2} \le \sqrt{2}$
	\item 
	\[
	\mathbb{E}\|X\|_2^2=\sum_{i=1}^{n}\mathbb{E}X_j^2=n
	\]
	\[
		\mathbb{P}\{\left| \|X\|_2^2 - n\right | \ge 0.01n \} =\mathbb{P} \left \{ \left |\sum_{i=1}^{n}(X_i^2-1)\right | \ge 0.01n \right \}
	\]
	As shown before, each $X_i$ is subgaussian, which makes $X_i^2-1$ sub-exponential for each $i$.
	
	Applying Bernstein's inequality
	\begin{align}\label{eq:1}
	\mathbb{P}\left \{ \left | \sum_{i=1}^{n}(X_i^2-1)\right | \ge 0.01n \right\} \le 2exp(-c \min(\frac{10^{-4}n^2}{\sigma^2},\frac{0.01n}{k}))
	\end{align}
	Where $$\sigma^2 = \sum_{i=1}^{n}\|X_i-1\|_{\psi_1}^2 \le nC$$ $$k=\max_{i\in[n]}\|X_i-1\|_{\psi_1}^2\le C$$ since all variables are sub-exponential.
	Substituting in (\ref{eq:1}), we have
	\[
	\mathbb{P}\{\left| \|X\|_2^2 - n\right | \le 0.01n \} \ge 1-2exp(-c^\prime n)
	\]
which implies that 
\[
0.99\sqrt{n}\le \|X\|_2 \le 1.01 \sqrt{n}
\]
with probability at least $1-2exp(-c^\prime n)$

\item Fix a pair $x_i$ and $x_j$. Normalizing $\frac{x_i-x_j}{\|x_i-x_j\|}$, we can apply the thin-shell we found in previous section, yielding
\begin{align*}
\mathbb{P}\left \{0.99\sqrt{n} \le \frac{\|Gx_i-Gx_j\|_2}{\|x_i-x_j\|_2} \le 1.01\sqrt{n} \right \} &=\\
\mathbb{P}\left \{0.99 \|x_i-x_j\|_2 \le \|{Tx_i-Tx_j}\|_2 \le 1.01 \|x_i-x_j\|_2 \right \} &\ge\\ 1-2exp(-cn)
\end{align*}
Applying union bound on all pairs, the probability of existing a pair not lying in the interval is bounded above by $2exp(2\log N-cn)$. Setting $n\ge \frac{4\log N}{c}$ would give us a high confidence.
	\end{enumerate}
	\qed
\end{bx}

\begin{bx}
	
	\begin{ex}\;
		\begin{enumerate}[label=(\alph*)]
			\item Check that the Gram matrix G of any system of vectors is positive semidefinite.
			\item Conversely, prove that any $n\times n$ positive semidefinite matrix $G$ is a Gram matrix
			of some system of vectors $v_1,\dots,v_n$ in $\mathbb{R}^n$ .
		\end{enumerate}
	\end{ex}
\tcblower
\begin{enumerate}[label=(\alph*)]
\item \label{it:a} Assume we have a set of vectors $v_1,\dots,v_n$ whose gram matrix is $G$.
\[
V = 
\begin{bmatrix}
v_1 \\
\vdots \\
v_n
\end{bmatrix}
\]
\[
G = 
\begin{bmatrix}
	\langle v_1,v_1\rangle & \dots & \langle v_1,v_n\rangle \\
	\vdots & \ddots&\vdots \\
	\langle v_n,v_1\rangle & \dots & \langle v_n,v_n\rangle
\end{bmatrix}
= VV^{\top}
\]
let $x$ be a non-zero vector. 
\[
x^{\top}Gx=x^{\top}VV^{\top}x = (V^{\top}x)^{\top}V^{\top}x = \langle V^{\top}x,V^{\top}x \rangle = \|V^{\top}x\|^2\ge 0
\]
where $\|.\|$ denotes the norm induced by the inner product.
\item According to the real spectral theorem, we can pick an orthonormal basis $v_1,\dots,v_n$ of $\mathbb{R}^n$ and write 
\[
G = V\Sigma V^{\top}
\]
where V is the column vector $V = [v_1,\dots,v_n]$ and $\Sigma$ is the diagonal matrix of eigenvalues.
expanding the relation, we get $G=\sum_{i=1}^{n}\lambda_iv_iv_i^{\top}$.
Setting $A=\sum_{i=1}^{n}\sqrt{\lambda_i}v_iv_i^{\top}$,
\[
A^2 = \sum_{i,j=1}^{n}\sqrt{\lambda_i\lambda_j}v_i\langle v_i,v_j \rangle v_j^{\top} = \sum_{i=1}^{n}\lambda_iv_iv_i^{\top} = G
\]
Note that the square root is real-valued since all eigenvalues are non-negative.
Further, it's easy to check that $A^2=AA^{\top}$, which is the gram matrix of the rows of $A$ as shown in part \ref{it:a}. So $G$ is the gram matrix of the rows of $A$.
\end{enumerate}
\qed
\end{bx}

\begin{bx}
	
	\begin{ex}
		Let $\theta$ be a random vector uniformly distributed on the unit circle $S^1 = \{x \in \mathbb{R}^2 :
			x_1^2 + x_2^2 = 1\}$. Prove that for any pair of vectors $u,v \in S^1$ , we have
			\[
			\mathbb{E}\mathrm{sign}(\langle u,\theta\rangle)\mathrm{sign}(\langle v,\theta\rangle) = \frac{2}{\pi}\mathrm{arcsin}(\langle u,v\rangle)
			\]
	\end{ex}
	\tcblower
	Setting $\angle(u,v)=\alpha$, we have $\alpha =\cos^{-1}(\langle u,v\rangle)$
	\begin{align*} 
		\mathbb{E}\mathrm{sign}(\langle u,\theta\rangle)\mathrm{sign}(\langle v,\theta\rangle) &= 1-\frac{2\alpha}{\pi} \\
		&= \frac{2}{\pi}\sin^{-1}(\langle u,v \rangle)
	\end{align*}
	\qed
\end{bx}
\end{document}